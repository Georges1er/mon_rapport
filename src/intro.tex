%% Copyright (C) 2008 Johan Oudinet <oudinet@lri.fr>
%%  
%% Permission is granted to make and distribute verbatim copies of
%% this manual provided the copyright notice and this permission notice
%% are preserved on all copies.
%%  
%% Permission is granted to process this file through TeX and print the
%% results, provided the printed document carries copying permission
%% notice identical to this one except for the removal of this paragraph
%% (this paragraph not being relevant to the printed manual).
%%  
%% Permission is granted to copy and distribute modified versions of this
%% manual under the conditions for verbatim copying, provided that the
%% entire resulting derived work is distributed under the terms of a 
%% permission notice identical to this one.
%%  
%% Permission is granted to copy and distribute translations of this manual
%% into another language, under the above conditions for modified versions,
%% except that this permission notice may be stated in a translation
%% approved by the Free Software Foundation
%%  
\chapter{Introduction}  

Du 13 Avril au 30 Septembre 2015, nous avons réalisé un stage de recherche au sein d’Orange Labs Issy-les-Moulineaux dans le département  Sécurité conformément au cahier de charges de l’Université Paris-Sud. 
C’est un stage de fin d’études qui permet à l’étudiant de s’imprégner des réalités de l’entreprise d’une part et d’autre part des aspects de la recherche afin 
de développer chez l’étudiant des aptitudes et attitudes en la matière.              
 
\section{Contexte et objectifs du stage}
\label{sec:contexte}

Le groupe Orange est né en 2003 suite au  rachat de la société Orange par l'opérateur principal français des Télécommunications,France Télécom. Orange ,implanté actuellement dans 29 pays du monde occupe le quatrième rang mondial des sociétés de Télécommunications.Orange Labs est la branche recherche et développement du groupe. Elle mène des projets de recherche pour améliorer la qualité des services fournit par Orange et également concevoir des produits innovants. Orange place le service au cœur des enjeux d'aujourd'hui et de demain.
Afin  d'assurer une politique de croissance durable, le groupe Orange a mis en place des stratégies de développement. En effet, bien que le groupe Orange soit spécialisé dans les télécommunications, ces dernières années Orange a diversifié ses offres en devenant fournisseur d'accès internet, de services cloud et même d'objets connectés. \newline{}
\newline
\par 
Par ailleurs,l'informatique dans les nuages ou cloud computing est un modèle  informatique permettant  l'accès à des ressources partagées au travers d'un réseau haut débit (Internet). Ce nouveau concept informatique a révolutionné l'industrie informatique.  En effet, le cloud permet de  mutualiser les ressources informatiques sur des serveurs géographiquement répartis, les utiliser à la demande et être facturés à l'utilisation. Ce qui permet de réduire les coûts de production et d'exploitation. En dépit de ces différentes facilités qu'offre le  cloud, la sécurité demeure  le principal  frein  à son  adoption.  Le cloud comme bien d'autres infrastructures tirant parti d'Internet, est également confronté à des attaques visant la disponibilité, l'intégrité et la confidentialité des données. Récemment, de nouvelles attaques  par  canaux auxiliaires  ont émergé dans le cloud. Ces attaques se basent sur des mécanismes indirects de compromission  tels que l'observation  et la mesure de l'activité des ressources partagées (cache, disque dur)  pour extirper des clés de chiffrement. \newline{}
\newline
\par 
Tout fournisseur de cloud comme Orange se doit d'assurer à ses clients la sécurité de ses plateformes contre ces attaques qui deviennent de plus en plus réalistes.
Face à cette situation et soucieux de la menace,  Orange Labs,  à travers  son département sécurité dont la mission  principale est  de veiller à la sureté et la sécurité des contenus offerts s’évertue à rechercher des mécanismes adéquats  et efficaces dans l’optique d’anticiper ces éventuelles menaces. C’est pourquoi,
Orange Labs a proposé ce stage de recherche dont le thème est : «Protection contre les attaques par canaux auxiliaires visant l’hyperviseur.». Les principaux objectifs du stage sont les suivants:
\begin{enumerate}
\item L'établissement d'une cartographie des contre-mesures existantes (au niveau matériel, système, ou applicatif) et les comparer (performance, sécurité, compatibilité) avec les clouds existants);
\item La proposition d'une nouvelle contremesure contre ces attaques;
\item La validation sur des plateformes cloud.
\end{enumerate}    

\section{Motivations et plan du rapport}    
\label{sec:plan} 

Un parcours succinct de la littérature montre que la première étude menée  sur les attaques par canaux auxiliaires dans le cloud a été réalisée dans \cite{Ris2009}. Cette étude a montré l’effectivité de ce type d' attaque sur les plateformes actuelles de cloud.  Un potentiel attaquant peut  en exploitant des  failles matérielles d’implémentation  déployer sa machine virtuelle(VM) sur une même plateforme qu’une victime cible et dérober des informations confidentielles en examinant par exemple le trafic réseau ou les frappes de clavier. A la suite de cette attaque, plusieurs autres attaques ont été réalisées \cite{Zhang2012} et \cite{Yuval}. Cependant  l’attaque la plus significative  est celle de Zhang et al. \cite{Zhang2012}  qui a permis de recouvrir la totalité d’une clé de chiffrement dans la bibliothèque de sécurité \cite{gnupg}.\newline{}

\par 
Plusieurs  contremesures ont été proposées tant par des chercheurs académiques que par  des chercheurs industriels permettant de se prémunir contre ces attaques. Ces moyens de lutte  sont regroupés en plusieurs catégories:
\begin{itemize} 
 \item  \textbf{Limitations du partage du cache}.  Ces contremesures visent à réduire le partage du cache. Ainsi, on peut citer \cite{STEALTHMEM} et \cite{chameleon}. Un autre type de contremesures  faisant partie de cette classe consiste à effectuer  \textbf{bruitage des données dans  le cache} (\cite{duppel}, \cite{Godefrey}) de telle sorte que l’attaquant n’arrive pas à distinguer les données présentes dans le cache
 \item \textbf{Contraintes de sécurité}. Dans cette catégorie, on tient compte des contraintes de sécurité pendant le placement des machine virtuelles. Ainsi  \cite{Lyon1} et \cite{Lyon2}  proposent des métriques et heuristiques de placement  qui permettent à des entités, respectivement, de quantifier la vulnérabilité de ces attaques et de  déployer ou  non  leurs machines sur des mêmes environnements  que des attaquants potentiels. 
 \item  \textbf{Obfuscation du temps}. Contrairement aux contremesures précédentes, cette contremesure  vise à obfusquer la mesure du temps des attaquants.  StopWatch \cite{StopWatch}  offusque le temps par la réplication des  machines virtuelles  en trois exemplaires  et en  utilisant   la valeur moyenne du temps des réplicas. TimeWarp \cite{TimeWarp} quant à lui ajoute des délais aléatoires aux mesures de temps.  
 \item  \textbf{Ordonnancement}. Ristenpart et al. \cite{Ristenpart} ont investigué l’approche de contremesure basée sur l’ordonnancement des machines virtuelles.  Avec un minimum de temps d’ordonnancement, on arrive à inhiber l’attaque.
\end{itemize} \newline{}

          
\newline{}
\par
Cependant, toutes ces contremesures ne sont spécifiques qu’à une seule couche d’abstraction de l’infrastructure cloud (couche virtuelle ou hyperviseur ou couche physique) ou à un type particulier d’application. Nous croyons qu’une meilleure sécurité nécessite d’entrevoir l’attaque dans sa globalité et non seulement à un seul niveau. De plus, nous pensons que les contremesures proposées doivent  être génériques et leur exécution doit pouvoir se faire de manière autonome. Un système  de sécurité autonome permet en effet, une mise en oeuvre plus flexible et plus efficiente  des poliques de sécurité d'une infrastructure répartie. Nous proposons donc d’étudier et d’évaluer l’approche innovante des contremesures multicouches qui nous semble être plus sécurisante. 
Ainsi, les principales contributions  de ce travail de recherche  sont les suivantes:
\begin{enumerate}
 \item La cartographie des différentes attaques par canaux auxiliaires et des contremesures existantes ;
 \item L’élaboration d’un nouvel algorithme de détection des attaques par canaux cachés auxiliaires dans le cloud;
 \item La proposition d'un Framework autonome de protection multicouche et un benchmark de ce Framework.
\end{enumerate}
  
Pour mener ce travail, nous présenteront d'abord dans la section suivante l’état de l’art des attaques par canaux auxiliaires et des contremesures proposées ainsi que les principaux concepts de notre étude. Dans une troisième section nous présenterons nos principales contributions. Nous  évaluerons ces contributions dans la section 4  et enfin nous conclurons.     


%%% Local Variables: 
%%% mode: latex
%%% TeX-master: "rapportM2R"
%%% End: 
