%% Copyright (C) 2008 Johan Oudinet <oudinet@lri.fr>
%%  
%% Permission is granted to make and distribute verbatim copies of
%% this manual provided the copyright notice and this permission notice
%% are preserved on all copies.
%%  
%% Permission is granted to process this file through TeX and print the
%% results, provided the printed document carries copying permission
%% notice identical to this one except for the removal of this paragraph
%% (this paragraph not being relevant to the printed manual).
%%  
%% Permission is granted to copy and distribute modified versions of this
%% manual under the conditions for verbatim copying, provided that the
%% entire resulting derived work is distributed under the terms of a 
%% permission notice identical to this one.
%%  
%% Permission is granted to copy and distribute translations of this manual
%% into another language, under the above conditions for modified versions,
%% except that this permission notice may be stated in a translation
%% approved by the Free Software Foundation
%%  
\chapter{Introduction}

\section{Contexte et objectifs du stage}
\label{sec:contexte}

Le présent rapport de stage résulte de 6 mois de travail réalisés au sein d'Orange Labs Issy-les-Moulineaux dans le département  Sécurité. Le groupe Orange est né en 2003 suite au  rachat de la société Orange par l'opérateur principal français des Télécommunications,France Télécom. Orange étant implanté actuellement dans 29 pays du monde occupe le quatrième rang mondial des sociétés de Télécommunications.Orange Labs est la branche R&D du groupe. Elle mène des projets de recherche pour améliorer la qualité des services fournit par Orange et également concevoir des produits innovants. Orange place le service au cœur des enjeux d'aujourd'hui et de demain.
Afin  d'assurer une politique de croissance durable, le groupe Orange a mis en place des stratégies de développement. En effet, bien que le groupe Orange soit spécialisé dans les télécommunications, ces dernières années Orange a diversifié ses offres en devenant fournisseur d'accès internet, de services cloud et même d'objets connectés. 
\par 
Par ailleurs,l'informatique dans les nuages ou cloud computing est un modèle  informatique permettant  l'accès à des ressources partagées au travers d'un réseau haut débit (Internet). Ce nouveau concept informatique a révolutionné l'industrie informatique.  En effet, le cloud permet de  mutualiser les ressources informatiques sur des serveurs géographiquement répartis, les utiliser à la demande et être facturés à l'utilisation. Ce qui permet de réduire les coûts de production et d'exploitation. En dépit de ces différentes facilités qu'offre le  cloud, la sécurité demeure  le principal  frein  à son  adoption.  Le cloud comme bien d'autres infrastructures tirant parti d'Internet, est également confronté à des attaques visant la disponibilité, l'intégrité et la confidentialité des données. Récemment, de nouvelles attaques  par  canaux auxiliaires  ont émergé dans le cloud. Ces attaques se basent sur des mécanismes indirects de compromission  tels que l'observation  et la mesure de l'activité des ressources partagées (cache, disque dur)  pour extirper des clés de chiffrement. 
\par 
Tout fournisseur de cloud comme Orange se doit d'assurer à ses clients la sécurité de ses plateformes contre ces attaques qui deviennent de plus en plus réalistes.
Ainsi,le département sécurité d'Orange Labs dont la mission  principale est  de veiller à la sureté et la sécurité des contenus offerts par Orange et d'anticiper les éventuelles menaces, s'est penché sur la question des attaques par canaux auxiliaires dans le cloud dans le cadre de ses activités de recherche. Le stage présent a alors été proposé.les principaux objectifs du stage sont :
L'établissement d'une cartographie des contre-mesures existantes (au niveau matériel, système, ou applicatif) et les comparez (performance, sécurité, compatibilité) avec les clouds existants.
La proposition d'une nouvelle contremesure contre ces attaques
La validation sur des plateformes cloud


\section{Eo adducta re per Isauriam}
\label{sec:isauriam}

Eo adducta re per Isauriam, rege Persarum bellis finitimis inligato
repellenteque a conlimitiis suis ferocissimas gentes, quae mente
quadam versabili hostiliter eum saepe incessunt et in nos arma
moventem aliquotiens iuvant, Nohodares quidam nomine e numero
optimatum, incursare Mesopotamiam quotiens copia dederit ordinatus,
explorabat nostra sollicite, si repperisset usquam locum vi subita
perrupturus.

Batnae municipium in Anthemusia conditum Macedonum manu priscorum ab
Euphrate flumine brevi spatio disparatur, refertum mercatoribus
opulentis, ubi annua sollemnitate prope Septembris initium mensis ad
nundinas magna promiscuae fortunae convenit multitudo ad commercanda
quae Indi mittunt et Seres aliaque plurima vehi terra marique
consueta.
  
%%% Local Variables: 
%%% mode: latex
%%% TeX-master: "rapportM2R"
%%% End: 
